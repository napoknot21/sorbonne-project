\documentclass[11pt,a4paper]{article}

% --------------------------------------------------
% Packages
% --------------------------------------------------
\usepackage[T1]{fontenc}
\usepackage[utf8]{inputenc}
\usepackage[french]{babel}

\usepackage{amsmath,amssymb}
\usepackage{graphicx}
\usepackage{hyperref}
\usepackage{geometry}
\usepackage{float}

\geometry{margin=2.5cm}

% --------------------------------------------------
% Title
% --------------------------------------------------
\title{Méthode des éléments finis 1D\\
Influence du maillage uniforme et géométrique}
\author{Charly MARTIN AVILA - 21503268}
\date{Décembre 2025}

\newpage


\begin{document}

\maketitle

% ==================================================
\section{Introduction}
% ==================================================

La méthode des éléments finis est une méthode de discrétisation largement utilisée
pour la résolution numérique des équations aux dérivées partielles.
Son efficacité dépend fortement de la régularité de la solution recherchée
et du choix du maillage utilisé pour la discrétisation du domaine.

Dans ce projet, on étudie un problème de Poisson unidimensionnel
dont la solution exacte présente une singularité locale.
L’objectif est d’analyser l’impact de cette singularité sur la convergence
de la méthode des éléments finis P1
et de montrer comment un choix approprié du maillage,
en particulier un maillage géométrique,
peut améliorer significativement la précision de l’approximation numérique.

% ==================================================
\section{Problème modèle}
% ==================================================

On considère le problème elliptique suivant :
\begin{equation}
\begin{cases}
- u''(x) = f(x), \quad x \in (0,1), \\
u(0) = u(1) = 0.
\end{cases}
\end{equation}

La solution exacte est donnée par :
\begin{equation}
u(x) = x^{3/4}(1-x).
\end{equation}

Cette fonction appartient à l’espace $H^1_0(0,1)$,
mais sa dérivée présente une singularité en $x=0$,
ce qui implique que $u \notin H^2(0,1)$.
Ce défaut de régularité est un élément central du projet,
car il influence directement les ordres de convergence
que l’on peut espérer obtenir avec les éléments finis P1.

% ==================================================
\section{Méthode des éléments finis P1}
% ==================================================

On cherche une approximation $u_h$ de la solution exacte $u$
dans l’espace des fonctions continues,
affines par morceaux sur un maillage de l’intervalle $(0,1)$,
et nulles aux bornes.
La formulation variationnelle du problème conduit à :
\[
\int_0^1 u'(x) v'(x)\,dx = \int_0^1 f(x)v(x)\,dx,
\quad \forall v \in H^1_0(0,1).
\]

La discrétisation par éléments finis P1 conduit à un système linéaire
$AU=b$, où $A$ est la matrice de rigidité et $b$ le vecteur de charge.
Le second membre est approché par une quadrature au point milieu.

Les erreurs numériques sont mesurées en norme énergie
$\|u-u_h\|_{H^1_0}$ et en norme $L^2$,
afin d’analyser finement l’influence du maillage.

% ==================================================
\section{Maillage uniforme : convergence (Question 5)}
% ==================================================

\begin{figure}[H]
\centering
\includegraphics[width=0.75\textwidth]{plots/q5_uniform_energy.png}
\caption{Erreur en norme énergie $\|u-u_h\|_{H^1_0}$ pour un maillage uniforme (échelle log-log).}
\label{fig:q5_uniform_energy}
\end{figure}

La figure~\ref{fig:q5_uniform_energy} montre la convergence de l’erreur en norme énergie
lorsque le nombre d’éléments $N$ augmente.
Cependant, le taux de convergence observé est inférieur à celui attendu
dans le cas d’une solution régulière.
Ce comportement est parfaitement cohérent avec la théorie :
la singularité de la dérivée de la solution exacte près de $x=0$
empêche l’approximation P1 sur un maillage uniforme
d’atteindre l’ordre de convergence optimal.

\begin{figure}[H]
\centering
\includegraphics[width=0.75\textwidth]{plots/q5_uniform_l2.png}
\caption{Erreur en norme $L^2$ pour un maillage uniforme (échelle log-log).}
\label{fig:q5_uniform_l2}
\end{figure}

La figure~\ref{fig:q5_uniform_l2} présente l’évolution de l’erreur en norme $L^2$.
La convergence est plus rapide qu’en norme énergie,
ce qui est conforme à l’analyse théorique des éléments finis.
Néanmoins, l’ordre reste limité par la régularité insuffisante de la solution,
confirmant que le maillage uniforme n’est pas optimal pour ce problème.

% ==================================================
\section{Maillages géométriques (Question 6)}
% ==================================================

\begin{figure}[H]
\centering
\includegraphics[width=0.85\textwidth]{plots/q6_meshes.png}
\caption{Maillages géométriques pour $N=50$ et différentes valeurs du paramètre $\alpha$.}
\label{fig:q6_meshes}
\end{figure}

La figure~\ref{fig:q6_meshes} illustre l’effet du paramètre $\alpha$
dans la construction d’un maillage géométrique.
Lorsque $\alpha>1$, les points sont concentrés près de l’origine,
zone où la solution présente une singularité.
Mathématiquement, ce choix de maillage vise à compenser la perte de régularité
en réduisant la taille des éléments dans les zones critiques,
où l’erreur locale d’approximation est dominante.

% ==================================================
\section{Convergence sur maillage géométrique (Question 7)}
% ==================================================

\begin{figure}[H]
\centering
\includegraphics[width=0.75\textwidth]{plots/q7_geo_energy.png}
\caption{Erreur en norme énergie pour un maillage géométrique avec $\alpha=4$.}
\label{fig:q7_geo_energy}
\end{figure}

La figure~\ref{fig:q7_geo_energy} montre une amélioration nette de la convergence
en norme énergie par rapport au maillage uniforme.
Le raffinement local du maillage permet de mieux approximer la solution
dans les zones où sa dérivée est singulière,
ce qui réduit l’erreur globale mesurée par la norme énergie.

\begin{figure}[H]
\centering
\includegraphics[width=0.75\textwidth]{plots/q7_geo_l2.png}
\caption{Erreur en norme $L^2$ pour un maillage géométrique avec $\alpha=4$.}
\label{fig:q7_geo_l2}
\end{figure}

La figure~\ref{fig:q7_geo_l2} confirme également une amélioration en norme $L^2$.
Bien que cette norme soit moins sensible aux variations locales de la dérivée,
l’adaptation du maillage améliore la qualité globale de l’approximation.

% ==================================================
\section{Choix du paramètre $\alpha$ à $N$ fixé (Question 8)}
% ==================================================

\begin{figure}[H]
\centering
\includegraphics[width=0.80\textwidth]{plots/q8_error_vs_alpha.png}
\caption{Erreur en norme énergie et en norme $L^2$ en fonction de $\alpha$ pour $N=50$.}
\label{fig:q8_alpha}
\end{figure}

La figure~\ref{fig:q8_alpha} montre clairement l’influence du paramètre $\alpha$
lorsque le nombre de points est fixé.
Pour des valeurs faibles de $\alpha$, le maillage reste trop uniforme
et ne capture pas correctement la singularité.
Lorsque $\alpha$ augmente, l’erreur diminue grâce au raffinement local.
Cependant, pour des valeurs trop grandes de $\alpha$,
la taille excessive des éléments loin de l’origine dégrade l’approximation globale.
Il existe donc une valeur optimale de $\alpha$,
résultant d’un compromis entre raffinement local et précision globale.

% ==================================================
\section{Comparaison des solutions à $N$ fixé (Question 9)}
% ==================================================

\begin{figure}[H]
\centering
\includegraphics[width=0.9\textwidth]{plots/q9_solutions.png}
\caption{Comparaison des solutions numériques pour $N=50$ :
maillage uniforme et géométrique ($\alpha=4$), comparées à la solution exacte.}
\label{fig:q9_solutions}
\end{figure}

La figure~\ref{fig:q9_solutions} montre que les deux solutions numériques
sont presque indiscernables visuellement.
Ce comportement est attendu, car pour un nombre d’éléments suffisant,
les deux méthodes convergent vers la solution exacte.
La différence entre les deux approches est principalement quantitative
et apparaît dans les normes d’erreur,
en particulier près de $x=0$ où la solution est peu régulière.

% ==================================================
\section{Conclusion}
% ==================================================

Ce projet met en évidence le lien étroit entre régularité de la solution,
choix du maillage et performance de la méthode des éléments finis.
Dans le cas d’une solution présentant une singularité locale,
le maillage uniforme conduit à une convergence sous-optimale.
L’utilisation d’un maillage géométrique permet d’adapter la discrétisation
aux propriétés analytiques de la solution
et d’améliorer significativement la précision numérique.
Ces résultats soulignent l’importance de l’analyse mathématique
dans la conception de méthodes numériques efficaces.

\end{document}
